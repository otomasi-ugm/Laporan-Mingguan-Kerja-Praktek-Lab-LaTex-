\documentclass{article}
\usepackage{graphicx}
\usepackage{xcolor}

\begin{document}
	
	
	\title{Format Laporan Kerja Praktik}
	\author{Disusun oleh Fahmizal, S.T., M.Sc.}
	\maketitle
	
	\begin{abstract}
		Ini adalah sebuah draft menggunakan LaTex untuk membuat laporan mingguan selama melaksanakan Kerja Praktik di Laboratorium Instrumentasi dan Kendali, Departemen Teknik Elektro dan Informatika, Sekolah Vokasi, Universitas Gadjah Mada. 
	\end{abstract}
	
	
	\section{Pendahuluan} 
	Pada sub bab ini akan dijelaskan bagaimana membuat sebuah equation di LaTex. Berikut contoh penulisan formula yang dihasilkan oleh LaTex.
	
	\begin{equation}
	\label{simple_equation}
	\alpha = \sqrt{ \beta }
	\end{equation}
	
	\begin{equation}
	\label{simple_equation}
	A=B^2
	\end{equation}
	
	\subsection{Sejarah Singkat}
	Penelitian ini diawali ketika penemuan formula ...
	
	\section{Metode}
	Metode yang digunakan pada penelitian ini menggunakan dua metode yang berbeda, yaitu:
	\begin{itemize}
		\item Metode 1
		\item Metode 2
		
	\end{itemize}
	
	\section{Gambar and Tabel}
	\paragraph{Posisi dari Gambar dan Tabel.} 	Berikut ini cara menuliskan sintax di LaTex untuk menampilkan sebuah Gambar.	``Gbr.~\ref{fig}'', Gambar dapat disingkat dengan Gbr.
	
	\begin{table}[htbp]
		\caption{Table Type Styles}
		\begin{center}
			\begin{tabular}{|c|c|c|c|}
				\hline
				\textbf{Table}&\multicolumn{3}{|c|}{\textbf{Table Column Head}} \\
				\cline{2-4} 
				\textbf{Head} & \textbf{\textit{Table column subhead}}& \textbf{\textit{Subhead}}& \textbf{\textit{Subhead}} \\
				\hline
				copy& More table copy$^{\mathrm{a}}$& &  \\
				\hline
				\multicolumn{4}{l}{$^{\mathrm{a}}$Sample of a Table footnote.}
			\end{tabular}
			\label{tab1}
		\end{center}
	\end{table}
	
	\begin{figure}[htbp]
		\centerline{\includegraphics{fig1.png}}
		\caption{Example of a figure caption.}
		\label{fig}
	\end{figure}
	
		\section{Kesimpulan}
		Ini adalah contoh sederhana penggunaan LaTex untuk tujuan penulisan laporan Mingguan kerja Praktik di Laboratorium Instrumentasi dan Kendali, Departemen Teknik Elektro dan Informatika, Sekolah Vokasi, Universitas Gadjah Mada.
	
\section*{Ucapan Terimakasih}
Kami ucapkan terimakasih kepada Departemen Teknik Elektro dan Informatika, Sekolah Vokasi, Universitas Gadjah Mada atas support yang diberikan.

\section*{Daftar Pustaka}
Laporan ini menggunakan stye citation IEEE format.

\begin{thebibliography}{00}
	
	\bibitem{b1} G. Eason, B. Noble, and I. N. Sneddon, ``On certain integrals of Lipschitz-Hankel type involving products of Bessel functions,'' Phil. Trans. Roy. Soc. London, vol. A247, pp. 529--551, April 1955.
	\bibitem{b2} J. Clerk Maxwell, A Treatise on Electricity and Magnetism, 3rd ed., vol. 2. Oxford: Clarendon, 1892, pp.68--73.
	\bibitem{b3} I. S. Jacobs and C. P. Bean, ``Fine particles, thin films and exchange anisotropy,'' in Magnetism, vol. III, G. T. Rado and H. Suhl, Eds. New York: Academic, 1963, pp. 271--350.
	\bibitem{b4} K. Elissa, ``Title of paper if known,'' unpublished.
	\bibitem{b5} R. Nicole, ``Title of paper with only first word capitalized,'' J. Name Stand. Abbrev., in press.
	\bibitem{b6} Y. Yorozu, M. Hirano, K. Oka, and Y. Tagawa, ``Electron spectroscopy studies on magneto-optical media and plastic substrate interface,'' IEEE Transl. J. Magn. Japan, vol. 2, pp. 740--741, August 1987 [Digests 9th Annual Conf. Magnetics Japan, p. 301, 1982].
	\bibitem{b7} M. Young, The Technical Writer's Handbook. Mill Valley, CA: University Science, 1989.
\end{thebibliography}
\vspace{12pt}
\color{red}
Silahkan mengunjungi web kami di otomasi.sv.ugm.ac.id.  
\end{document}
